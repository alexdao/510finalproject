\documentclass{article}
\title{CS510 Project Proposal: Detailed Diagnostic Distributed File System}
\date{\today}
\author{Alex Dao, Jiawei Zhang, and Danny Oh}

\usepackage{amsmath}
\usepackage[margin=1.5in]{geometry}

\begin{document}
\maketitle

\section{Problem}
Recently, there has been a large shift towards storing large amounts of data in remote data centers, oftentimes distributed across many locations and hardware. Many current distributed file systems offer a layer of abstraction between a user and the distribution of the data, so that the user only experiences a file system like that on a Unix machine. However, in some cases, it would be useful for the user to actively see specifically which workers hold the data within a file system and how many concurrent users are requesting access to that file. 

Such a file system would provide important diagnostic information for job scheduling. If a user had a batch of jobs to be completed, they could be ordered in such a way that minimized the number of conflicts with concurrent users, based on the data provided by the DDDFS. Additionally, such data could be collected on the client-side for analysis on the usage the data within the file system, which could prove useful.

\section{Methodology} 
Why is this approach new?

\section{Related Work}

\section{Research Plan Timeline}

\section{Resources}

\end{document}