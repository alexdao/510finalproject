\documentclass{article}
\setlength{\parskip}{\baselineskip}%

\title{CS510 Project Proposal: Detailed Diagnostic Distributed File System}
\date{\today}
\author{Alex Dao, Jiawei Zhang, and Danny Oh}

\usepackage{amsmath}
\usepackage[margin=1.5in]{geometry}
\usepackage{hyperref}

\begin{document}
\maketitle

\section{Problem}
Recently, there has been a large shift towards storing large amounts of data in remote data centers, oftentimes distributed across many locations and hardware. Many current distributed file systems offer a layer of abstraction between a user and the distribution of the data, so that the user only experiences a file system like that on a Unix machine. However, in some cases, it would be useful for the user to actively see specifically which workers hold the data within a file system and how many concurrent users are requesting access to that file. 

Such a file system would provide important diagnostic information for job scheduling. If a user had a batch of jobs to be completed, they could be ordered in such a way that minimized the number of conflicts with concurrent users, based on the data provided by the DDDFS. Additionally, such data could be collected on the client-side for analysis on the usage the data within the file system, which could prove useful.

A sub-problem of distributed file systems are their performance, especially with regards to variable-size files. The Google File System\textsuperscript{[1]}, for example, is highly optimized for large files. Using such a system with small files is both space-inefficient and slow. 

\section{Methodology} 

The distributed file system is envisioned as a number of closely-located servers operating on Linux run on commodity hardware. Similar to the Google File System, requests for data are routed through a single master server that serves only metadata. 

To optimize the file system for variable-size files, DDDFS takes the counter-intuitive approach of storing large files on the same machine and distributing small files across as many machines as possible. We argue that distributing the Disk-IO for small files across multiple machines will increase the speed of disk-accesses to use the network more optimally. We make the assumption that the commodity hardware is utilizing spinning-drive disks, which often have small-file write-performance of under 8 MB/s, which is roughly on-par with the 72 Mbps average network speed in the United States. 

Adding detailed diagnostic data to a distributed file system involves the following main steps:

\begin{enumerate}
\item Addition of access logs (metadata) stored on a logging server (separate from master, which only serves requests)
	\begin{enumerate}
	\item The logging server continually queries slave servers and compiles the metadata
	\item There is no need for master to communicate to the logging server as master communicates to the slave servers (who serve data), which the logging server queries. 
	\end{enumerate}
\item Addition of file logs for every file stored on each slave server
	\begin{enumerate}
	\item These file logs contains every modification and access to the files, including the timestamp, user, and write/read
	\item These file logs do not contain any data as DDDFS is not meant to be a version control system
	\end{enumerate}
\end{enumerate}

Additionally, the logging server also generates access data for the distributed file system's own use and has the ability to move small files between slave servers (and update master server metadata) in order to optimize small-file access performance. 

These are two new ideas introduced by DDDFS. First is the distribution of small files across machines to decrease net latency when a number of small files are requested. Second is the addition of the logging server whose singular task is to serve metadata about file accesses to clients to improve their file-access scheduling. 

\section{Related Work}

\begin{enumerate}
\item \href{https://www.eecs.harvard.edu/cs261/papers/sandberg85.pdf}{Design and Implementation of the Sun Network Filesystem}: One of the first implementations of remote access filesystems. 
\item \href{http://www.cs.duke.edu/courses/spring16/compsci510/resources/papers/gfs-sosp2003.pdf}{The Google Filesystem}: Provides a good starting point for the architecture of the file system. 
\item \href{https://www.eecs.harvard.edu/cs261/papers/afs.pdf}{Scale and Performance in a Distributed File System}: Another implementation and analysis of a DFS. Also includes an interesting idea about aggregation of files into volumes that we may be able to build upon.
\item \href{https://www.eecs.harvard.edu/cs261/papers/sandberg85.pdf}{Fast, Cloud-scale Block Storage}: An example of current developments of storage on the cloud scale.
\item \href{http://storageconference.us/2010/Papers/MSST/Shvachko.pdf}{The Hadoop Distributed File System}: File System designed to store large datasets. Characterized by the partitioning of data and computations in parallel.
\end{enumerate}

\section{Research Plan Timeline}
\subsubsection*{Friday, April 8, 2016}
\begin{enumerate}
\item Benchmarking existing distributed file system performance for comparison. This involves benchmarking disk-io and network speeds. 
\end{enumerate}

\subsubsection*{Wednesday, April 13, 2016}
\begin{enumerate}
\item Design mechanism for communication by client to master, master to slaves, and logging to slaves. 
\item Design small-file distribution algorithm
\end{enumerate}

\subsubsection*{Monday, April 18, 2016}
\begin{enumerate}
\item Implement communications
\item Implement file distribution algorithm
\end{enumerate}

\subsubsection*{Wednesday, April 20, 2016}
\begin{enumerate}
\item Consolidation of implementations
\end{enumerate}

\section{Necessary Resources}

\begin{enumerate}
\item Implementations of a variety of existing file systems
\item A platform that can mimic a distributed file server. Ideally, given an interface to such a server, we would implement our algorithm over it.
\end{enumerate}

\section{References}
[1] S. Ghemawat, H. Gobioff, S. Leung. ``The Google file system,'' In Proc.
of ACM Symposium on Operating Systems Principles, Lake George,
NY, Oct 2003, pp 29–43.
\end{document}